\documentclass[12pt]{article}
\usepackage{amsmath}
\usepackage{amssymb}
\usepackage{graphicx}
\usepackage{tabulary}


\begin{document}

\title{Product Development Specification}%replace X with the appropriate number
\author{William Harrington, Jake Heath, Saroj Bardewa,\\ Shan Quinney, Michael Mathis\\ \\ %replace with your name
ECE411} %if necessary, replace with your course title
 
\maketitle
 \small
\begin{description} 
	\item[Customer requirements] \hfill \\ \\
		The Portland State Aerospace Society (PSAS) needs a device that can be used for command, control, communications between a \textbf{CubeSat} and a \textbf{ground station}.
		A \textbf{CubeSat} is a miniature satellite that can be used for scientific research in space. It has a volume of one liter (10cm cube), and has a mass of no more than 1.33kg. 
		A \textbf{ground station} is a radio station located on earth that is designed for communication with spacecraft.
		Development of such a device is not trivial as it will need to eventually be "space ready" and will require multiple phases of development. \hfill \\ \\
		The first phase of development will involve finding and de-risking a microcontroller that is capable of handling command, control, and communications. 
		Therefore, the device at this point will be a communications module "breakout board" that will allow easy access to the features of the microcontroller like UART, I2C, SPI, etc. 
		\hfill \\ \\
		\newpage
		\textbf{Must}
		\begin{itemize}
			\item{Have access to UART}
			\item{Have access to I2C}
			\item{Have access to SPI}
			\item{Be capable of RF communication with another device}
		\end{itemize}
		\textbf{Should}
		\begin{itemize}
			\item{Be battery powered}
			\item{If battery powered, utilize USB for recharging}
		\end{itemize}
		\textbf{May}
		\begin{itemize}
			\item{Be capable of long distance communication}
		\end{itemize}
\end{description}

\end{document}